\documentclass[]{book}
\usepackage{lmodern}
\usepackage{amssymb,amsmath}
\usepackage{ifxetex,ifluatex}
\usepackage{fixltx2e} % provides \textsubscript
\ifnum 0\ifxetex 1\fi\ifluatex 1\fi=0 % if pdftex
  \usepackage[T1]{fontenc}
  \usepackage[utf8]{inputenc}
\else % if luatex or xelatex
  \ifxetex
    \usepackage{mathspec}
  \else
    \usepackage{fontspec}
  \fi
  \defaultfontfeatures{Ligatures=TeX,Scale=MatchLowercase}
\fi
% use upquote if available, for straight quotes in verbatim environments
\IfFileExists{upquote.sty}{\usepackage{upquote}}{}
% use microtype if available
\IfFileExists{microtype.sty}{%
\usepackage{microtype}
\UseMicrotypeSet[protrusion]{basicmath} % disable protrusion for tt fonts
}{}
\usepackage[margin=1in]{geometry}
\usepackage{hyperref}
\hypersetup{unicode=true,
            pdftitle={Asset Pricing},
            pdfauthor={Christoffer Jul Elben},
            pdfborder={0 0 0},
            breaklinks=true}
\urlstyle{same}  % don't use monospace font for urls
\usepackage{natbib}
\bibliographystyle{apalike}
\usepackage{longtable,booktabs}
\usepackage{graphicx,grffile}
\makeatletter
\def\maxwidth{\ifdim\Gin@nat@width>\linewidth\linewidth\else\Gin@nat@width\fi}
\def\maxheight{\ifdim\Gin@nat@height>\textheight\textheight\else\Gin@nat@height\fi}
\makeatother
% Scale images if necessary, so that they will not overflow the page
% margins by default, and it is still possible to overwrite the defaults
% using explicit options in \includegraphics[width, height, ...]{}
\setkeys{Gin}{width=\maxwidth,height=\maxheight,keepaspectratio}
\IfFileExists{parskip.sty}{%
\usepackage{parskip}
}{% else
\setlength{\parindent}{0pt}
\setlength{\parskip}{6pt plus 2pt minus 1pt}
}
\setlength{\emergencystretch}{3em}  % prevent overfull lines
\providecommand{\tightlist}{%
  \setlength{\itemsep}{0pt}\setlength{\parskip}{0pt}}
\setcounter{secnumdepth}{5}
% Redefines (sub)paragraphs to behave more like sections
\ifx\paragraph\undefined\else
\let\oldparagraph\paragraph
\renewcommand{\paragraph}[1]{\oldparagraph{#1}\mbox{}}
\fi
\ifx\subparagraph\undefined\else
\let\oldsubparagraph\subparagraph
\renewcommand{\subparagraph}[1]{\oldsubparagraph{#1}\mbox{}}
\fi

%%% Use protect on footnotes to avoid problems with footnotes in titles
\let\rmarkdownfootnote\footnote%
\def\footnote{\protect\rmarkdownfootnote}

%%% Change title format to be more compact
\usepackage{titling}

% Create subtitle command for use in maketitle
\newcommand{\subtitle}[1]{
  \posttitle{
    \begin{center}\large#1\end{center}
    }
}

\setlength{\droptitle}{-2em}
  \title{Asset Pricing}
  \pretitle{\vspace{\droptitle}\centering\huge}
  \posttitle{\par}
  \author{Christoffer Jul Elben}
  \preauthor{\centering\large\emph}
  \postauthor{\par}
  \predate{\centering\large\emph}
  \postdate{\par}
  \date{2017-08-06}

\usepackage{fancyhdr}
\usepackage{lastpage}
\usepackage{adjustbox}
\usepackage{subcaption}
\usepackage{color}
\usepackage{xcolor}
\usepackage{graphicx}
\usepackage{textcomp}
\usepackage{lipsum}
\usepackage{afterpage}
\usepackage{xcolor}
\usepackage{pagecolor}
\usepackage{float}
\usepackage{cmbright}
\usepackage{dblfloatfix}  
\usepackage[framemethod=tikz]{mdframed}
\usepackage{fix-cm}
\graphicspath{{./figures/}}
\definecolor{darkblue}{rgb}{0.1,0.0,0.5}

\fancypagestyle{style1}{
\fancyhf{}
\fancyhead[L]{\includegraphics[height=1.25cm]{logo_aarhus_university}}
\fancyhead[R]{\includegraphics[height=1.25cm]{logo_aarhus_bss}}
\fancyfoot[C]{\textcopyright 2017 Christoffer Jul Elben}
\renewcommand{\headrulewidth}{0.4pt}
}

\fancypagestyle{style2}{
\fancyhf{}
\fancyhead[L]{\includegraphics[height=1.25cm]{logo_aarhus_university}}
\fancyhead[R]{\includegraphics[height=1.25cm]{logo_aarhus_bss}}
\fancyfoot[C]{\textcopyright 2017 Christoffer Jul Elben}
\fancyfoot[R]{\thepage\ of \pageref{last_page}}
\renewcommand{\headrulewidth}{0.4pt}
}

\fancypagestyle{style3}{
\fancyhf{}
\fancyhead[L]{\includegraphics[height=1.25cm]{logo_aarhus_university}}
\fancyhead[R]{\includegraphics[height=1.25cm]{logo_aarhus_bss}}
\fancyfoot[C]{\textcopyright 2017 Christoffer Jul Elben}
\fancyfoot[R]{\thepage\ of \pageref{last_page_appendix}}
\renewcommand{\headrulewidth}{0.4pt}
}

\usepackage{amsthm}
\newtheorem{theorem}{Theorem}[chapter]
\newtheorem{lemma}{Lemma}[chapter]
\theoremstyle{definition}
\newtheorem{definition}{Definition}[chapter]
\newtheorem{corollary}{Corollary}[chapter]
\newtheorem{proposition}{Proposition}[chapter]
\theoremstyle{definition}
\newtheorem{example}{Example}[chapter]
\theoremstyle{remark}
\newtheorem*{remark}{Remark}
\begin{document}
\maketitle

{
\setcounter{tocdepth}{1}
\tableofcontents
}
\chapter{Preface}\label{preface}

\[2+2=4\]

Hello world \citep[p.~123]{book_danthine} \citep[p.~123]{book_veronesi}.

\chapter{Danthine \& Donaldson: Chapter 2: The Challenges of Asset
Pricing: A Road
Map}\label{danthine-donaldson-chapter-2-the-challenges-of-asset-pricing-a-road-map}

Text

\section{Pre-lecture notes}\label{pre-lecture-notes}

Text

\section{Lecture notes}\label{lecture-notes}

Text

\section{Exercises}\label{exercises}

\chapter{Danthine \& Donaldson: Chapter 3: Making Choices in Risky
Situations}\label{danthine-donaldson-chapter-3-making-choices-in-risky-situations}

Text

\section{Pre-lecture notes}\label{pre-lecture-notes-1}

Text

\section{Lecture notes}\label{lecture-notes-1}

Text

\subsection{Exercise 3.1}\label{exercise-3.1}

\emph{Utility function: Under certainty, any increasing monotone
transformation of a utility function is also a utility function
representing the same preferences. Under uncertainty, we must restrict
this statement to linear transformations if we are to keep the same
preference representation. Give a mathematical as well as an economic
interpretation for this.} \citep[p.4]{exercises_danthine}

\emph{Check it with this example. Assume an initial utility function
attributes the following values to three perspectives:}
\citep[p.4]{exercises_danthine}

\[B \ \ \ u\left(B\right)=100\] \[M \ \ \ u\left(M\right)=10\]
\[P \ \ \ u\left(P\right)=50\]

\begin{enumerate}
\def\labelenumi{\alph{enumi}.}
\item
  \emph{Check that with this initial utility function, the lottery
  \(L=\left(B,M,0.50\right)\succ P\).} \citep[p.4]{exercises_danthine}
\item
  \emph{The proposed transformations are
  \(f\left(x\right)=a+bx, a\geq 0,b>0\) and
  \(g\left(x\right)=ln\left(x\right)\). Check that under
  \(f, L\succ P\).} \citep[p.5]{exercises_danthine}
\end{enumerate}

\subsection{Exercise 3.2}\label{exercise-3.2}

\emph{Lotteries: Discuss the equivalence between
\(\left(x,z,\pi\right)\) and
\(\left(x,y,\pi+\left(1-\pi\right)\tau\right)\) when
\(z=\left(x,y,\tau\right)\). Can you think of circumstances under which
they would not be viewed as equal?} \citep[p.5]{exercises_danthine}

\subsection{Exercise 3.3}\label{exercise-3.3}

\emph{Intertemporal consumption: Consider a two-date (one-period)
economy and an agent with utility function over consumption:}
\citep[p.5]{exercises_danthine}

\[U\left(c\right)=\frac{c^{1-\gamma}}{1-\gamma}\]

\emph{at each period. Define the intertemporal utility function as
\(V\left(c_1,c_2\right)=U\left(c_1\right)+U\left(c_2\right)\). Show (try
it mathematically) that the agent will always prefer a smooth
consumption stream to a more variable one with the same mean, that is,}
\citep[p.5]{exercises_danthine}

\[U\left(\bar{c}\right)+U\left(\bar{c}\right)>U\left(c_1\right)+U\left(c_2\right)\]
\[If \ \bar{c}=\frac{c_1+c_2}{2}\]

\chapter{Danthine \& Donaldson: Chapter 4: Measuring Risk and Risk
Aversion}\label{danthine-donaldson-chapter-4-measuring-risk-and-risk-aversion}

Text

\section{Pre-lecture notes}\label{pre-lecture-notes-2}

Text

\section{Lecture notes}\label{lecture-notes-2}

Text

\section{Exercises}\label{exercises-1}

\subsection{Exercise No. 4.1}\label{exercise-no.-4.1}

\emph{Risk aversion: Consider the following utility functions (defined
over wealth Y):} \citep[p.5]{exercises_danthine}

\[\left(1\right) \ U\left(Y\right)=-\frac{1}{Y}\]
\[\left(2\right) \ U\left(Y\right)=ln \ Y\]
\[\left(3\right) \ U\left(Y\right)=-Y^{-\gamma}\]
\[\left(4\right) \ U\left(Y\right)=-exp\left(-\gamma Y\right)\]
\[\left(5\right) \ U\left(Y\right)=\frac{Y^\gamma}{\gamma}\]
\[\left(6\right) \ U\left(Y\right)=\alpha Y-\beta Y^2\]

\begin{enumerate}
\def\labelenumi{\alph{enumi}.}
\item
  \emph{Check that they are well behaved \(\left(U'>0,U''<0\right)\) or
  state restrictions on the parameters so that they are {[}utility
  functions (1) -- (6){]}. For utility function (6), take positive
  \(\alpha\) and \(\beta\), and give the range of wealth over which the
  utility function is well behaved.} \citep[p.6]{exercises_danthine}
\item
  \emph{Compute the absolute and relative risk-aversion coefficients.}
  \citep[p.6]{exercises_danthine}
\item
  \emph{What is the effect of the parameter \(\gamma\) (when relevant)?}
  \citep[p.6]{exercises_danthine}
\item
  \emph{Classify the functions as increasing/decreasing risk-aversion
  utility functions (both absolute and relative).}
  \citep[p.6]{exercises_danthine}
\end{enumerate}

\subsection{Exercise 4.2}\label{exercise-4.2}

\emph{Certainty equivalent:} \citep[p.6]{exercises_danthine}

\[\left(1\right) \ U=-\frac{1}{Y}\] \[\left(2\right) \ U=ln \ Y\]

\[\left(3\right) \ U=\frac{Y^\gamma}{\gamma}\]

\emph{Consider the lottery \(L_1=\left(50,000;10,000;0.50\right)\).
Determine the lottery \(L_2=\left(x;0;1\right)\) that makes an agent
indifferent to lottery \(L_1\) with utility functions (1), (2), and (3)
as defined. For utility function (3), use \(\gamma=\{0.25,0.75\}\). What
is the effect of changing the value of \(\gamma\)? Comment on your
results using the notions of risk aversion and certainty equivalent.}
\citep[p.6]{exercises_danthine}

\subsection{Exercise 4.3}\label{exercise-4.3}

\emph{Risk premium: A businesswoman runs a firm worth CHF 100,000. She
faces some risk of having a fire that would reduce her net worth
according to the following three states, \(i=1,2,3\), each with
probability \(\pi\left(i\right)\) (Scenario A).}
\citep[p.7]{exercises_danthine}

\begin{center}\includegraphics[width=150px]{figures/matrix} \end{center}

\emph{Of course, in state 3, nothing detrimental happens, and her
business retains its value of CHF 100,000.}
\citep[p.7]{exercises_danthine}

\begin{enumerate}
\def\labelenumi{\alph{enumi}.}
\item
  \emph{What is the maximum amount she will pay for insurance if she has
  a logarithmic utility function over final wealth? (Note: The insurance
  pays CHF 99,999 in the first case; CHF 50,000 in the second; and
  nothing in the third.)} \citep[p.7]{exercises_danthine}
\item
  \emph{Do the calculations with the following alternative
  probabilities:} \citep[p.7]{exercises_danthine}
\end{enumerate}

\begin{center}\includegraphics[width=150px]{figures/matrix} \end{center}

\emph{Is the outcome (the comparative change in the premium) a surprise?
Why?} \citep[p.7]{exercises_danthine}

\subsection{Exercise 4.4}\label{exercise-4.4}

\emph{Consider two investments \(A\) and \(B\). Suppose that their
returns, \(\tilde{r}_A\) and \(\tilde{r}_B\), are such that
\(\tilde{r}_A=\tilde{r}_B+\vartheta\), where \(\vartheta\) is a
nonnegative random variable. Show that \(A\) FSD \(B\).}
\citep[p.7]{exercises_danthine}

\subsection{Exercise 4.5}\label{exercise-4.5}

\emph{Four-partquestion:} \citep[p.7]{exercises_danthine}

\begin{enumerate}
\def\labelenumi{\alph{enumi}.}
\item
  \emph{Explain intuitively the concept of first-order stochastic
  dominance.} \citep[p.7]{exercises_danthine}
\item
  \emph{Explain intuitively the concept of second-order stochastic
  dominance.} \citep[p.8]{exercises_danthine}
\item
  \emph{Explain intuitively the mean variance criterion.}
  \citep[p.8]{exercises_danthine}
\item
  \emph{You are offered the following two investment opportunities.}
  \citep[p.8]{exercises_danthine}
\end{enumerate}

\begin{center}\includegraphics[width=150px]{figures/matrix} \end{center}

\emph{apply concepts a--d. Illustrate the comparison with a graph.}
\citep[p.8]{exercises_danthine}

\subsection{Exercise 4.6}\label{exercise-4.6}

\emph{An individual (operating in perfect capital markets) with a zero
initial wealth, and the utility function
\(U\left(Y\right)=Y\frac{1}{2}\) is confronted with the gamble
\(\left(16,4;\frac{1}{2}\right)\).} \citep[p.8]{exercises_danthine}

\begin{enumerate}
\def\labelenumi{\alph{enumi}.}
\item
  \emph{What is his certainty equivalent for this gamble?}
  \citep[p.8]{exercises_danthine}
\item
  \emph{If there was an insurance policy that, together with the
  original gamble, would guarantee him the expected payoff of the
  gamble, what is the maximum premium he would be willing to pay for
  it?} \citep[p.8]{exercises_danthine}
\item
  \emph{What is the minimum required increase (the probability premium)
  in the probability of the high-payoff state so that he will not be
  willing to pay any premium for such an insurance policy? (Note that
  the insurance policy still pays the expected payoff of the unmodified
  gamble)} \citep[p.8]{exercises_danthine}
\item
  \emph{Now assume that he is confronted with the gamble
  \(\left(36,16;\frac{1}{2}\right)\). Calculate the certainty
  equivalent, the insurance premium, and the probability premium for
  this case as well. Explain what is going on, and why?}
  \citep[p.8]{exercises_danthine}
\end{enumerate}

\subsection{Exercise 4.7}\label{exercise-4.7}

\emph{Refer to Table 4.2. Suppose the return data for investment 3 was
as follows. Is it still the case} \citep[p.9]{exercises_danthine}

\begin{center}\includegraphics[width=150px]{figures/matrix} \end{center}

\emph{that investment 3 SSD investment 4?}
\citep[p.9]{exercises_danthine}

\subsection{Exercise 4.8}\label{exercise-4.8}

\emph{Consider two investments with the following characteristics:}
\citep[p.9]{exercises_danthine}

\begin{center}\includegraphics[width=150px]{figures/matrix} \end{center}

\begin{enumerate}
\def\labelenumi{\alph{enumi}.}
\item
  \emph{Is there state-by-state dominance between these two
  investments?} \citep[p.9]{exercises_danthine}
\item
  \emph{Is there FSD between these two investments?}
  \citep[p.9]{exercises_danthine}
\end{enumerate}

\subsection{Exercise 4.9}\label{exercise-4.9}

\emph{If you are exposed to a 50/50 probability of gaining or losing CHF
1'000 and an insurance that removes the risk costs CHF 500, at what
level of wealth will you be indifferent between taking the gamble or
paying the insurance? That is, what is your certainty equivalent wealth
for this gamble? Assume that your utility function is
\(U\left(Y\right)=-1/Y\). What would the solution be if the utility
function were logarithmic?} \citep[p.9]{exercises_danthine}

\subsection{Exercise 4.10}\label{exercise-4.10}

\emph{Assume that you have a logarithmic utility function on wealth
\(U\left(Y\right)=ln \ Y\) and that you are faced with a 50/50
probability of winning or losing CHF 1'000. How much will you pay to
avoid this risk if your current level of wealth is CHF 10'000? How much
would you pay if your level of wealth is CHF 1'000'000? Did you expect
that the premium you were willing to pay would increase/decrease? Why?}
\citep[p.9]{exercises_danthine}

\subsection{Exercise 4.11}\label{exercise-4.11}

\emph{Assume that security returns are normally distributed. Compare
portfolios A and B, using both first and second-order stochastic
dominance:} \citep[p.10]{exercises_danthine}

\(Case 1 \\ \sigma_a>\sigma_b \\ E_a=E_b\)
\(Case 2 \\ \sigma_a>=\sigma_b \\ E_a>E_b\)
\(Case 3 \\ \sigma_a><\sigma_b \\ E_a<E_b\) HELP HELP HELP

\subsection{Exercise 4.12}\label{exercise-4.12}

\emph{An agent faces a risky situation in which he can lose some amount
of money with probabilities given in the following table:}
\citep[p.10]{exercises_danthine}

\begin{center}\includegraphics[width=150px]{figures/matrix} \end{center}

\emph{This agent has a utility function of wealth of the form}
\citep[p.10]{exercises_danthine}

\[U\left(Y\right)=\frac{Y^{1-\gamma}}{1-\gamma}+2\]

\emph{His initial wealth level is 10000 and his \(\gamma\) s equal to
1.2.} \citep[p.10]{exercises_danthine}

\begin{enumerate}
\def\labelenumi{\alph{enumi}.}
\item
  \emph{Calculate the certainty equivalent of this prospect for this
  agent. Calculate the risk premium. What would be the certainty
  equivalent of this agent if he would be risk neutral?}
  \citep[p.10]{exercises_danthine}
\item
  \emph{Describe the risk premium of an agent whose utility function of
  wealth has the form implied by the following properties:
  \(U'\left(Y\right)>0\) and \(U''\left(Y\right)>0\)}
  \citep[p.10]{exercises_danthine}
\end{enumerate}

\subsection{Exercise 4.13}\label{exercise-4.13}

\emph{An agent with a logarithmic utility function of wealth tries to
maximize his expected utility. He faces a situation in which he will
incur a loss of \(L\) with probability \(\pi\). He has the possibility
to insure against this loss. The insurance premium depends on the extent
of the coverage. The amount covered is denoted by \(h\) and the price of
the insurance per unit of coverage is \(p\) (hence the amount he has to
spend on the insurance will be \(hp\)).}
\citep[p.10]{exercises_danthine}

\begin{enumerate}
\def\labelenumi{\alph{enumi}.}
\item
  \emph{Calculate the amount of coverage h demanded by agent as a
  function of his wealth level \(Y\), the loss \(L\), the probability
  \(\pi\) and the price of the insurance \(p\).}
  \citep[p.10]{exercises_danthine}
\item
  \emph{What is the expected gain of an insurance company offering such
  a contract ?} \citep[p.10]{exercises_danthine}
\item
  \emph{If there is perfect competition in the insurance market (zero
  profit), what price p will the insurance company set?}
  \citep[p.10]{exercises_danthine}
\item
  \emph{What amount of insurance will the agent buy at the price
  calculated under c. What is the influence of the form of the utility
  function?} \citep[p.11]{exercises_danthine}
\end{enumerate}

\bibliography{references.bib}


\end{document}
